% USN logo på norsk og engelsk 
\begin{figure}[!ht]
  \centering
  \includegraphics[width=0.8\textwidth]{USN_logo}
  \caption{Norwegian (bokmål) variant of the USN logo}
  \label{fig:usn-logo}
\end{figure}
\lipsum[4]
\begin{figure}[!ht]
  \centering
  \includegraphics[width=0.8\textwidth]{USN_logo_en}
  \caption{English variant of the USN logo.}
  \label{fig:usn-logo-en}
\end{figure}
\lipsum

% Formler 
\begin{equation}
  e = m c^2
\end{equation}

%Theory
\section{Maxwell's Equations}
\label{sec:theory}
\indent The differential forms of Maxwell's equations as found by Heaviside, while completely valid, are now considered somewhat archaic, and have been replaced by the more useful (equivalent) integral forms. Each law is named according to the person(s) who originally discovered the connections represented by the equation. Here are the four equations:
\begin{eqnarray}
  \text{Gauss' law for electricity:}& \displaystyle \oint{\vec{E}\cdot\mathrm{d}\vec{A}}&=\frac{Q_{enc}}{\epsilon_0}\\
  \text{Gauss' law for magnetism:}& \displaystyle \oint{\vec{B}\cdot\mathrm{d}\vec{A}}&=0\\
  \text{Faraday's law:}& \displaystyle\oint{\vec{E}\cdot\mathrm{d}\vec{s}}&=-\frac{\emph{d}\phi_b}{\mathrm{d}t}\\
  \text{Ampere-Maxwell law:}& \displaystyle\oint{\vec{B}\cdot\mathrm{d}\vec{s}}&=\mu_0\epsilon_0\frac{\emph{d}\phi_e}{\mathrm{d}t}+\mu_0 i_{enc}
\end{eqnarray}
Note: $\oint$ is used to specify a closed loop integral, also known as a line integral. It simply means that in the calculations, we must go all the way around the loop; we can't stop part way through or the equations won't be valid.

\section{Mathematical model}
\label{sec:mathmodel}
\lipsum[8]
\begin{table}[!ht]
  \caption{The different number systems}
  \centering
  \begin{tabular}{|r|l|}
    \hline
    7C0 & hexadecimal \\
    3700 & octal \\ \cline{2-2}
    11111000000 & binary \\
    \hline \hline
    1984 & decimal \\
    \hline
  \end{tabular}
\end{table}

\lipsum[4]

\begin{table}[!ht]
 \caption{The weather forecast}
  \centering
   \begin{tabular}{ | l | l | l | p{5cm} |}
    \hline
    Day & Min Temp & Max Temp & Summary \\ \hline
    Monday & 11C & 22C & A clear day with lots of sunshine.
    However, the strong breeze will bring down the temperatures. \\ \hline
    Tuesday & 9C & 19C & Cloudy with rain, across many northern regions. Clear spells
    across most of Scotland and Northern Ireland,
    but rain reaching the far northwest. \\ \hline
    Wednesday & 10C & 21C & Rain will still linger for the morning.
    Conditions will improve by early afternoon and continue
    throughout the evening. \\
    \hline
    \end{tabular}
\end{table}
\lipsum{100-150}