%% sample template file for a MSc Thesis
%% The default is with two sided setup:
\documentclass[%
% oneside,    %% uncomment for onesided layout
% project,    %% uncomment not thesis but project report
% nosummary   %% uncomment if no summary page should be generated
]{USN-MSc}

% --- Bibliography setup ---
%%% default is the "ieee" style
\usepackage[style=ieee, sorting=none]{biblatex}
%%% If you want to use "author-year" style
%%% where `\cite{Foo2011}` generates "Foo et al. (2011)"
%%% and   `\parentcite{Foo2011}` generates "(Foo et al. 2011)"
%%% then comment the line above and use
%\usepackage[style=authoryear]{biblatex}
%%% or
%%% if you want to use "alphabetic" style then use
%%% where `cite[Foo2011]` generates "[Foo11]"
%%% then comment the line above and use
%\usepackage[style=alphabetic]{biblatex}
%%% instead.
%% load the bib file:
\addbibresource{thesis.bib}

\usepackage{lipsum} % just for providing fill text used in this template

% --- general setup ---
%% Please fill in the following parameters:
\newcommand{\mytitle}{%
%% title:
Transient and long-term power system stability with Modelica
}

\newcommand{\mysubtitle}{%
%% master programme (for thesis only)
%% uncomment the appropriate one:
Electrical Power Engineering
%Energy and Environmental Technology
%Industrial IT and Automation
%Process Technology
}

\newcommand{\mykeywords}{%
%% keywords (for thesis only):
<keyword one, keyword two, \ldots>
}

\newcommand{\myauthor}{%
%% author(thesis) or group code (project):
Gunhild Marie Grimstvedt
}

\newcommand{\myparticipants}{
%% group participants (for project only)
<First participant>\\
<Second participant>\\
<Third participant>\\
<Fourth participant>
}

\newcommand{\supervisor}{%
%% supervisor:
Dietmar Winkler
}
\begin{document}

% --- title page setup ---
\USNtitlepage%
%% Please provide the following information:
%% #1 optional figure (set to {} if not wanted)
{%
  {\normalsize <optional figure>}
   \includegraphics[draft,width=\textwidth]{USN_logo_en}}
%% #2 confidential?:
{yes} % anything other then {yes} means open to public
%% #3 Project partner:
{<Project partner>}
%% #4 Summary:
{%
\lipsum[6-7]
}

\chapter*{Preface}
\label{sec:preface}
\addcontentsline{toc}{chapter}{Preface}
\lipsum[1-3]
\bigskip
Porsgrunn, \today

\myauthor %% for thesis
%\myparticipants %% for project


%% table of contents
\tableofcontents
\addcontentsline{toc}{chapter}{\contentsname}

\listoffigures % out-comment if unwanted
\addcontentsline{toc}{section}{\listfigurename}

\listoftables  % out-comment if unwanted
\addcontentsline{toc}{section}{\listtablename}

\chapter*{Nomenclature}
\label{sec:nomenclature}
bla

\begin{longtable}{ll}
  \textbf{Symbol} & \textbf{Explanation}\endhead\\
  A/D	& Analogue-Digital-Converter \\
  CMR	& Common Mode Rejection \\
  foo	& Foo \\
  bar 	& Bar
\end{longtable}

\chapter{Introduction}
\label{ch:intro}
\input{Theses/Ch1:Introduction/Introduction.tex}

\chapter{Theory}
\label{ch:theory}
\input{Theses/Ch2:Theory/Theory.tex}

\chapter{Requirements}
\label{ch:requirements}
\input{Theses/Ch3:Requirements/Requirements.tex}

\chapter{Simulations}
\label{ch:simulations}
\input{Theses/Ch4:Simulations/Simulations.tex}

\chapter{Discussion}
\label{ch:discussion}
\input{Theses/Ch5:Discussion/Discussion.tex}

\chapter{Conclusion}
\label{ch:conclusion}
\input{Theses/Ch6:Conclusion/Conclusion.tex}

% A dummy command that causes all bibliographyentries to be displayed
% even though there were not cited in the document. Used for demonstration
% purposes only in this template file.
~\nocite{*}

\cleardoublepage

% The bibliography should be displayed here...
\printbibliography[heading=bibintoc]
% You rather like to call the bibliography "References"? Then use this instead:
%\printbibliography[heading=bibintoc, title={References}]


\appendix
%\renewcommand{\appendixname}{Paper} %% So we get 'Paper X' displayed instead


\chapter[Short Title of Paper A]{Title of Paper A (probably very long and therefore not good to have in the header)}
\label{paper-a}

\paragraph{Note}
Since some papers tend to have a rather long title it is good to provide the optional short title which then will be displayed in the table of contents and header instead of the long original title.
On the openening page of the chapter the orginal \emph{long} title will be displayed.\bigskip

\emph{Short descriptive text of paper follows here.}\bigskip

The paper itself needs to be included in the published form as PDF on the next pages.
This can be done using the \texttt{pdfpages} package by adding the command:

\begin{verbatim}
\includepdf{pages=-,openright}{Filename}
\end{verbatim}

You can ommit the \texttt{.pdf} when specifying the \texttt{Filename}. Also you should include always include the option \texttt{openright} since it would look strange to have the paper starting at the back of the cover page.

There are more options like only adding specific pages:
\begin{verbatim}
\includepdf{pages=2-6,openright}{Filename.pdf}
\end{verbatim}

For more options see Appendix~\ref{paper-b} where the most important pages of the \texttt{pdfpages} manual were inlcuded using \texttt{pdfpages}.


%%% Command to include a PDF file directly including all pages:


\chapter[Short Title of Paper B]{Title of Paper B}
\label{paper-b}
Short descriptive text of paper follows here.

Here we included the first five pages of the \texttt{pdfpages} manual itself.

\includepdf[pages=1-5,openright]{fig/pdfpages}

\end{document}

%%% Local Variables:
%%% mode: latex
%%% TeX-master: t
%%% End:
