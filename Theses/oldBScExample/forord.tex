\documentclass[Visionprosjekt.tex]{subfiles} 
\NormalTopp
\thispagestyle{fancy}{\rhead{\toppfont Forord}}
\begin{document} 

%%%%%%%%%%%%%%%%%%%%%%%%%%%%%%%%%%%%%%%%%%%%%%%%%%%%%%%%%%%%%%%%%%%%%%%%
% Forord, NB! Ingen section her!
%%%%%%%%%%%%%%%%%%%%%%%%%%%%%%%%%%%%%%%%%%%%%%%%%%%%%%%%%%%%%%%%%%%%%%%%

Prosjektet er utført av en studentgruppe bestående av André Skare Berg, Espen Løkseth, Kristian Torsvik og Matias Wilhelmsen. Oppgaven er gjennomført i 6. semester ved \HiT, under utdanningen
for Informatikk og automatisering, Y-VEI ved Institutt for Elektro, IT og Kybernetikk. Fremdriftsplanen
er vedlagt som \refv{ved:framdrift}, WBS er \refv{ved:wbs} og oppgavetekst er  \refv{ved:oppgavetekst}.


  
Forsidebildet er hentet fra The Heart of Innovation, lokalisert 23.11.2010 på adressen  \url{http://www.ideachampions.com/weblogs/archives/2011/01/50_awesome_quot_1.shtml}.


Rapporten krever at leseren har grunnleggende kunnskaper om elektro, automasjons\-teknikk og PLS-pro\-gramm\-ering. Alle elektriske skjemaer og diagrammer følger norsk standard for elektroteknisk dokumentasjon \cite{nek144}. Se \refv{ved:skjema}.


Følgende dataverktøy har blitt brukt i hovedprosjektet:
\begin{itemize}
     \item \LaTeXe{} ved hjelp av MiK\TeX{} og \TeX nicCenter
	\item Autodesk AutoCAD Electrical 
     \item Microsoft Excel, PowerPoint, Visio og Project
     \item Siemens Simatic Step 7 og WinCC
     \item DVT FrameWork
\end{itemize}



Dette er den første studentrapporten  på bachelornivå som offisielt er skrevet i \LaTeX{} ved \HiT. 
Espen Løkseth har selv utformet \LaTeX-malen basert på \cite{wordmal}. 
Dette inkluderer også ${\mathrm{B{\scriptstyle{IB}} \! T\!_{\displaystyle E} \! X}}$-layouten, basert på \cite{litteraturliste}.

\vfill
Porsgrunn, \today.
\blanklines{3}


\begin{center}
    \begin{tabular}{ll} 
    \makebox[50mm]{\hrulefill}  \hspace{40mm}   & \makebox[50mm]{\hrulefill} \\
    André Skare Berg                & \Espen  \\[18mm]   
    \makebox[50mm]{\hrulefill}  \hspace{40mm}  & \makebox[50mm]{\hrulefill} \\
    Kristian Torsvik                & Matias Wilhelmsen     
    \end{tabular} 
\end{center}


\vspace{10mm}




\end{document}