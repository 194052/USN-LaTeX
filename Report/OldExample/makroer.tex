
% Fra Matte III

\newcommand{\Espen}{Espen L\o kseth} 
\newcommand{\FTint}{\ensuremath{\int_{C}{\vec{F}\cdot\vec{T}ds}}}
\newcommand{\Greens}{\ensuremath{\varointctrclockwise_{C}{Pdx+Qdy}=\iint_{R}{\left(\frac{\partial Q}{\partial x}-\frac{\partial P}{\partial y}\right)dA}}}
\newcommand{\parDiff}[2]{\ensuremath{\frac{\partial #1}{\partial #2 }}}
\newcommand{\dparDiff}[2]{\ensuremath{\dfrac{\partial #1}{\partial #2 }}}
\newcommand{\abs}[1]{\ensuremath{\left| #1 \right|}}

\newcommand{\mat}[2][rrrrrrrrrrrrrrrrrrrrrrrrrrrrrrrrrrrrrrrrrrrrrrrrrrr]{\left[\begin{array}{#1}#2 \\ \end{array} \right]}
\newcommand{\detM}[2][rrrrrrrrrrrrrrrrrrrrrrrrrrrrrrrrrrrrrrrrrrrrrrrrrrr]{\left|
\begin{array}{#1}#2 \\ 
\end{array} \right|}
\newcommand{\tekst}[1]{\mathrm{#1}}
\newcommand{\hatt}{\ensuremath{\wedge}}
\newcommand{\linjeR}{\ensuremath{\int_{C}{\vec{F}\,d\vec{r}}}}
\newcommand{\dr}{d\vec{r}}
\definecolor{mblue}{rgb}{0,0.08,0.45}
\definecolor{mgreen}{rgb}{0,0.6,0}
\newcommand{\green}[1]{\textcolor{green}{#1}}
\newcommand{\red}[1]{\textcolor{red}{#1}}
\newcommand{\blue}[1]{\textcolor{blue}{#1}}
\newcommand{\maple}[1]{

		\noindent\newline\textit{Maple:}\\
		\texttt{#1} \newline
}
\makeatletter
\newcommand{\romertall}[1]{\romannumeral #1}
\newcommand{\Romertall}[1]{\expandafter\@slowromancap\romannumeral #1@}
\makeatother
\newcommand{\storboks}[1]{\fbox{\parbox{\textwidth}{#1}}\vspace{3mm}}


\newcommand{\blankLinje}{\\ \newline}
\newcommand{\blanklinje}{\\ \newline}
\newcounter{cteller} 
\newcommand{\blanklines}[1]{
\forloop{cteller}{-1}{\value{cteller} < #1}{\noindent \newline}}
\newcommand{\blanklinjen}[1]{\blanklines{#1}}
\newcommand{\blankLinjeN}[1]{\blanklines{#1}}
\newcommand{\blankline}{\blanklines{1}}










% Fra Statistikk

\newcommand{\snitt}{\ensuremath{\cap}}
\newcommand{\union}{\ensuremath{\cup}}
\newcommand{\p}[1]{\ensuremath{P\!\left( #1 \right)}}
\newcommand{\ptekst}[1]{\ensuremath{P\!\left(\text{\dq #1\dq}\right)}}

\newcommand{\classpad}[1]{

		\noindent\newline\textit{ClassPad:}\\
		\texttt{#1} %\newline
}
\newcommand{\dsum}{\ensuremath{\displaystyle\sum}}
\newcommand{\dint}{\ensuremath{\displaystyle\int}}
\newcolumntype{x}[1]{>{\raggedright}p{#1}}
%\newcolumntype{x}[1]{>{\raggedleft}p{#1}}

\newcommand{\losning}{\noindent\newline
        \textit{Løsning:}\\}
        
\newtimeformat{tidsformatTall}{\twodigit{\THEHOUR}.\twodigit{\THEMINUTE}}
\newdateformat{datoformatTall}{\twodigit{\THEDAY}.\twodigit{\THEMONTH}.\THEYEAR}

\newcommand{\talldato}{\datoformatTall\today}
\newcommand{\tallklokke}{\tidsformatTall}

\newcommand{\sistKompilert}{\footnotesize\sffamily
    Kompilert \talldato{ } kl. \tallklokke \normalfont\normalsize}

%\newcommand{\N}{\ensuremath{\text{N}}}
\newcommand{\N}[1]{\ensuremath{\text{N}\!\left(#1\right)}}
\newcommand{\Bin}[1]{\ensuremath{\text{Bin}\!\left(#1\right)}}
\newcommand{\Hyp}[1]{\ensuremath{\text{Hyp}\!\left(#1\right)}}
\newcommand{\Poisson}[1]{\ensuremath{\text{Poissin}\!\left(#1\right)}}
\newcommand{\Geo}[1]{\ensuremath{\text{Geo}\!\left(#1\right)}}





% Fra kybernetikk

\newcommand{\Mat}[2][ccccccccccccccccccccccccccccccccccccccccccccc]{\left[\begin{array}{#1}#2 \\ \end{array}\right]}
\newcommand{\sq}{\textquotesingle} %For Maple og MATLAB!!!
\newcommand{\bs}{\textbackslash{}} 


%Diverse
\newcommand{\pref}[1]{\hyperref[#1]{(\ref*{#1})}}
\newcommand{\vfantom}[1]{\vphantom{\rule{0pt}{#1}}} 


% Fra matte 4
\newcommand{\Lap}[1]{\ensuremath{\mathcal{L}\left\{#1\right\}}}
\newcommand{\Fou}[1]{\ensuremath{\mathcal{F}\left\{#1\right\}}}




% Fra Industielle programmeringssystemer
% \mbox{} hindrer f.eks. at figur 2-1 blir splittet i "-"
\newcommand{\reff}[1]{\hyperref[#1] {figur   \mbox{\ref*{#1}}}}
\newcommand{\refb}[1]{\hyperref[#1] {figur   \mbox{\ref*{#1}}}}
\newcommand{\reffo}[1]{\hyperref[#1]{formel  \mbox{(\ref*{#1})}}}
\newcommand{\reft}[1]{\hyperref[#1] {tabell  \mbox{\ref*{#1}}}}
\newcommand{\refv}[1]{\hyperref[#1] {ved\-legg \ref*{#1}}}

\newcommand{\refF}[1]{\hyperref[#1] {Figur   \mbox{\ref*{#1}}}}
\newcommand{\refB}[1]{\hyperref[#1] {Figur   \mbox{\ref*{#1}}}}
\newcommand{\refFo}[1]{\hyperref[#1]{Formel  \mbox{(\ref*{#1})}}}
\newcommand{\refT}[1]{\hyperref[#1] {Tabell  \mbox{\ref*{#1}}}}
\newcommand{\refV}[1]{\hyperref[#1] {Ved\-legg \ref*{#1}}}

%\newcommand{\refp}[1]{\hyperref[#1]{\mbox{(\ref*{#1})}}}


\newcommand{\C}{\,\ensuremath{^\circ}C }

%
%\newcommand{\forfatter}[1]{#1.}
%\newcommand{\publiseringsdato}[1]{(#1).}
%\newcommand{\tittel}[1]{\emph{#1.}}
%\newcommand{\stedForlag}[1]{#1.}
%\newcommand{\lokaliseringsdato}[1]{Lokalisert #1:}
%


\newcommand{\Referanser}[1]{\clearpage
\let\a\refname
\renewcommand*\refname{\MakeUppercase{\a}}
\thispagestyle{fancy}{
\rhead{\toppfont\a}}
\phantomsection
\addcontentsline{toc}{section}{\a}
}

\newcommand{\Vedlegg}[1]{\clearpage
\thispagestyle{fancy}{
\rhead{\toppfont Vedlegg}}
\phantomsection
\addcontentsline{toc}{section}{Vedlegg}
\section*{Vedlegg}}


\newcommand{\NormalTopp}{
\setlength{\textheight}{\normalTeksthoyde}
\setlength{\headheight}{\normalToppteksthoyde}
\renewcommand{\sectionmark}[1]{
\markboth{\thesection \ ##1}{}}
\renewcommand{\footrulewidth}{0.4pt}
\thispagestyle{fancy}{\lhead{\toppfont\HiT}
\chead{}\cfoot{}\rhead{\toppfont\leftmark}
\lfoot{\toppfont\Gruppe}\rfoot{\toppfont\thepage}}
}


